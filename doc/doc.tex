\documentclass{article}

\title{CS3031 - Project 2}
\author{Paolo Moloney - 16325409}

\usepackage{listings}
\usepackage{color}
\usepackage{setspace}
\usepackage{graphicx}

\definecolor{Code}{rgb}{0,0,0}
\definecolor{Decorators}{rgb}{0.5,0.5,0.5}
\definecolor{Numbers}{rgb}{0.5,0,0}
\definecolor{MatchingBrackets}{rgb}{0.25,0.5,0.5}
\definecolor{Keywords}{rgb}{0,0,1}
\definecolor{self}{rgb}{0,0,0}
\definecolor{Strings}{rgb}{0,0.63,0}
\definecolor{Comments}{rgb}{0,0.63,1}
\definecolor{Backquotes}{rgb}{0,0,0}
\definecolor{Classname}{rgb}{0,0,0}
\definecolor{FunctionName}{rgb}{0,0,0}
\definecolor{Operators}{rgb}{0,0,0}
\definecolor{Background}{rgb}{0.98,0.98,0.98}

\lstdefinelanguage{Python}{
	numbers=left,
	numberstyle=\footnotesize,
	numbersep=1em,
       xleftmargin=1em,
       framextopmargin=2em,
       framexbottommargin=2em,
       showspaces=false,
       showtabs=false,
       showstringspaces=false,
	frame=l,
	tabsize=4,
	% Basic
	basicstyle=\ttfamily\small\setstretch{1},
	backgroundcolor=\color{Background},
	% Comments
	commentstyle=\color{Comments}\slshape,
	% Strings
	stringstyle=\color{Strings},
	morecomment=[s][\color{Strings}]{"""}{"""},
	morecomment=[s][\color{Strings}]{'''}{'''},
	% keywords
	morekeywords={import,from,class,def,for,while,if,is,in,elif,else,not,and,or,print,break,continue,return,True,False,None,access,as,,del,except,exec,finally,global,import,lambda,pass,print,raise,try,assert},
	keywordstyle={\color{Keywords}\bfseries},
	% additional keywords
	morekeywords={[2]@invariant,pylab,numpy,np,scipy},
	keywordstyle={[2]\color{Decorators}\slshape},
	emph={self},
	emphstyle={\color{self}\slshape},
	%
}
\begin{document}
\maketitle
\newpage
\tableofcontents
\newpage
\section{Specification}
	The objective of this project is to implement a secure cloud storage application with the following features:
	\begin{enumerate}
		\item Secures all files uploaded to the cloud, such that only users in the 'Secure Cloud Storage Group' can access them.
		\item Key management system for users.
		\item Add and remove users from the 'Secure Cloud Storage Group'.
	\end{enumerate}
	\newpage
\section{Implementation}
	I chose to implement the application in Python, using the Google Drive API and the following modules:
	\begin{description}
		\item[PyDrive]
			Wrapper library of google-api-python-client
		\item[cryptography]
			Provides high-level and low-level interfaces to common cryptographic algorithms
	\end{description}
	\subsection{Admin Mode}
		When the \texttt{admin.py} script is run, it first searches for a symmetric key in keys/key.txt. If none is found then it generates a new key and stores it there. This key is encrypted using RSA, providing an additional layer of security. The files in the cloud are stored in a folder on Google Drive. All these files are encryptred using this symmetric key. \\
		The admin has a number of management commands available:
		\begin{description}
			\item[enc]
				Pulls all the files from the Drive and encrypts using the symmetric key in keys/key.txt, then reuploads them
			\item[dec]
				Similar to enc, but decrypts instead of encrypting
			\item[lf]
				Lists all the files in the Drive
			\item[addu]
				Prompts the admin to enter a username to add to the 'Secure Cloud Storage Group', then adds this username to group/ and generates a private key with RSA, storing it in group/user/privateKey.txt, used to decrypt the symmetric key
			\item[rmvu]
				Prompts the admin to enter a username to remove from the 'Secure Cloud Storage Group', deleting the user and the corresponding private key from group/
			\item[lu]
				Lists all the users in the 'Secure Cloud Storage Group'
			\item[q]
				Terminates the execution of the program
		\end{description}
	\subsection{User Mode}
		Running \texttt{user.py} is similar to admin mode, but with less privileges. The commands available to users are:
		\begin{description}
			\item[lf]
				Lists all the files in the Drive
			\item[op]
				Prompts the user for the filename to view, then pulls and decrypts the file and prints its contents to the console
			\item[up]
				Prompts the user for the path to the file to upload \emph{(Not working)}
		\end{description}
	\subsection{Authentication}
		Authorization and authentication are handled by PyDrive. The credentials are found in client\_secrets.json, which is generated when setting up the Google Drive API project.
\section{Code}
	\subsection{admin.py}
	\lstinputlisting[language=Python]{../admin.py}
	\newpage
	\subsection{user.py}
	\lstinputlisting[language=Python]{../user.py}
\end{document}
